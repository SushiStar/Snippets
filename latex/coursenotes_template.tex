%%%%%%%%%%%%%%%%%%%%%%%% Packages %%%%%%%%%%%%%%%%%%%%%%%%%%%
\documentclass[letterpaper,11pt]{article}

% General usage: fonts, pics, etc.
\usepackage[margin=1in]{geometry}
\usepackage{times}
\usepackage{comment}
\usepackage{hyperref}

% math packages
\usepackage{mathtools}

% algorithms
\usepackage{algorithm}
\usepackage{algpseudocode}
\usepackage{algorithmicx}

%%%%%%%%%%%%%%%%%%%%%%%% Commands %%%%%%%%%%%%%%%%%%%%%%%%%%%%
\newcommand{\coursetopic}{Lecture Title}


%%%%%%%%%%%%%%%%%%%%%%%% Documents %%%%%%%%%%%%%%%%%%%%%%%%%%%
\begin{document}
\begin{center}{\huge \bfseries \coursetopic}\end{center}  % title
\begin{flushright} \today\end{flushright}                 % date




\end{document}




%%%%%%%%%%%%%%%%%%% Template for pseudocode %%%%%%%%%%%%%%%%%%
\begin{algorithm}
    \caption{Euclid’s algorithm}
    \label{euclid}
    \begin{algorithmic}[1] % The number tells where the line numbering should start
        \Procedure{Euclid}{$a,b$} \Comment{The g.c.d. of a and b}
            \State $r\gets a \bmod b$
            \While{$r\not=0$} \Comment{We have the answer if r is 0}
                \State $a \gets b$
                \State $b \gets r$
                \State $r \gets a \bmod b$
            \EndWhile\label{euclidendwhile}
            \State \textbf{return} $b$\Comment{The gcd is b}
        \EndProcedure
    \end{algorithmic}
\end{algorithm}

