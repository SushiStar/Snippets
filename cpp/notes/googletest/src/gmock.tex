%%%%%%%%%%%%%%%%%%%%%%%% Packages %%%%%%%%%%%%%%%%%%%%%%%%%%%
\documentclass[letterpaper,11pt]{report}

% General usage: fonts, pics, etc.
\usepackage[margin=.60in]{geometry}
\usepackage[stable]{footmisc}
\usepackage{times}
\usepackage{comment}
\usepackage{hyperref}
\usepackage{xcolor}
\usepackage{graphicx}

% math packages
\usepackage{mathtools}
\usepackage{eucal}
\usepackage{amssymb}

% algorithms
\usepackage{algorithm}
\usepackage{algpseudocode}
\usepackage{algorithmicx}
\usepackage{listings}

%%%%%%%%%%%%%%%%%%%%%%%% Commands %%%%%%%%%%%%%%%%%%%%%%%%%%%%
\newcommand{\coursetopic}{gMock}
\newcommand{\thedate}{Jan 25, 2021}

% set the default code style
\lstset{
    basicstyle=\footnotesize,
    frame=single, % draw a frame at the top and bottom of the code block
    tabsize=4, % tab space width
    showstringspaces=false, % don't mark spaces in strings
    % numbers=left, % display line numbers on the left
    commentstyle=\color{gray}, % comment color
    keywordstyle=\color{blue}, % keyword color
    stringstyle=\color{red} % string color
}

%%%%%%%%%%%%%%%%%%%%%%%% Documents %%%%%%%%%%%%%%%%%%%%%%%%%%%
\begin{document}
\begin{center}{\huge \bfseries \coursetopic}\end{center}  % title
\begin{flushright}\thedate\end{flushright} % date

\section*{What is gMock?}
When you write a prototype or test, often it's not feasible or wise to rely on real objects 
entirely.\\
A \textbf{mock object} implements the same interface as a real object (so it can be used as one),
but lets you specify at run time how it will be used and what it should do (which methods will be
called? in which order? how many times? with what arguments? what will they return?)

\textbf{Fake objects} have working implementations, but usually take some shortcut(perhaps to make 
the operations less expensive), which makes them not suitable for production.

\textbf{Mocks} are objects pre-programmed with \textit{expectations},
 which form a specification of the calls they are expected to receive.

Use gMock when:
\begin{itemize}
    \item You are stuck with a sub-optimal design and wish you had done more prototyping before it
        was too late, but prototyping in \texttt{C++} is by no means ``rapid''.
    \item Your tests are slow as they depend on too many libraries or use expensive resources.
    \item Your tests are brittle as some resources are unreliable.
    \item You want to test how your code handle failure, but it's not easy to cause one.
    \item You need to make sure your module interacts with other modules in the right way, but it's
        hard to observe the interaction; therefore you resort to observing the side effects at the
        end of the action, but it's awkward at best,
    \item You want to "mock out" your dependencies, except that they don't have mock implementations
        yet; and, frankly, you aren't thrilled by some of those hand written mocks.
\end{itemize}

\section*{Creating Mock Classes}
Mock classes are defined as normal classes, using the \texttt{MOCK\_METHOD} macro to genrate mocked
methods.

\end{document}
