%%%%%%%%%%%%%%%%%%%%%%%% Packages %%%%%%%%%%%%%%%%%%%%%%%%%%%
\documentclass[letterpaper,11pt]{report}

% General usage: fonts, pics, etc.
\usepackage[margin=.60in]{geometry}
\usepackage[stable]{footmisc}
\usepackage{times}
\usepackage{comment}
\usepackage{hyperref}
\usepackage{xcolor}
\usepackage{graphicx}

% math packages
\usepackage{mathtools}
\usepackage{eucal}
\usepackage{amssymb}

% algorithms
\usepackage{algorithm}
\usepackage{algpseudocode}
\usepackage{algorithmicx}
\usepackage{listings}

%%%%%%%%%%%%%%%%%%%%%%%% Commands %%%%%%%%%%%%%%%%%%%%%%%%%%%%
\newcommand{\coursetopic}{Googletest}
\newcommand{\thedate}{Jan 25, 2021}

% set the default code style
\lstset{
    basicstyle=\footnotesize,
    frame=single, % draw a frame at the top and bottom of the code block
    tabsize=4, % tab space width
    showstringspaces=false, % don't mark spaces in strings
    % numbers=left, % display line numbers on the left
    commentstyle=\color{gray}, % comment color
    keywordstyle=\color{blue}, % keyword color
    stringstyle=\color{red} % string color
}

%%%%%%%%%%%%%%%%%%%%%%%% Documents %%%%%%%%%%%%%%%%%%%%%%%%%%%
\begin{document}
\begin{center}{\huge \bfseries \coursetopic}\end{center}  % title
\begin{flushright}\thedate\end{flushright} % date

\section*{Thinkings on test}
\subsection*{Googletest}
\textit{googletest} helps you to write better \textit{C++} tests.
\subsection*{Good tests}
\begin{itemize}
    \item Tests should be independent and repeatable.
    \item Tests should be well organized and reflect the structure of the tested code.
    \item Tests should be portable and reusable. (\textit{platform-neutral})
    \item Tests should be fast.
    \item When tests fail, they should provide as much information about the problem as possible.
\end{itemize}

\section*{Basic Concepts}
\begin{itemize}
    \item When using googletest, we start by writing \textit{assertions}.
    \item An assertion's result can be \textbf{success}, \textbf{nonfatal failure}, \textbf{fatal failure},
if a fatal failure occurs, it aborts the current function; otherwise, the program continues normally.
    \item A \textit{test suite} contains one or many tests. You should group your tests into test suites that reflect 
the sturcture of the code. When multiple tests in a test suite need to share common objects and subroutines,
you can put them into a test fixture class.
    \item A test program can contain multiple test suites.
\end{itemize}

\section*{Assertions}
When an assertion fails, googletest prints the assertion's source file and line number location,
 along with a failure message.
\begin{itemize}
    \item \texttt{ASSERT\_*} versions generate fatal failures when they fail and abort current function.
    \item \texttt{EXPECT\_*} versions geenrate nonfatal failures, which don't abort current function.
    \item To provide a custom failure message, simply stream it to the macro using the \texttt{<<} operator.
\end{itemize}
\subsection*{Basic Assertion}
\begin{center}
    \begin{tabular}{| c | c | c |}
        \hline
        \textbf{Fatal Assertion}           & \textbf{Nonfatal Assertion}       & \textbf{Verifies}     \\ \hline 
        \texttt{ASSERT\_TRUE(condition)}   & \texttt{EXPECT\_TRUE(condition)}  & \texttt{condition} is true \\ \hline
        \texttt{ASSERT\_FALSE(condition)}  & \texttt{EXPECT\_FALSE(condition)} & \texttt{condition} is false \\ \hline
    \end{tabular}
\end{center}

\subsection*{Binary Comparison}
\begin{center}
    \begin{tabular}{| c | c | c |}
        \hline
        \textbf{Fatal Assertion}         & \textbf{Nonfatal Assertion}     & \textbf{Verifies}     \\ \hline 
        \texttt{ASSERT\_EQ(val1, val2)}  & \texttt{EXPECT\_EQ(val1, val2)} & \texttt{val1 == val2} \\ \hline
        \texttt{ASSERT\_NE(val1, val2)}  & \texttt{EXPECT\_NE(val1, val2)} & \texttt{val1 != val2} \\ \hline
        \texttt{ASSERT\_LT(val1, val2)}  & \texttt{EXPECT\_LT(val1, val2)} & \texttt{val1 < val2}  \\ \hline
        \texttt{ASSERT\_LE(val1, val2)}  & \texttt{EXPECT\_LE(val1, val2)} & \texttt{val1 <= val2} \\ \hline
        \texttt{ASSERT\_GT(val1, val2)}  & \texttt{EXPECT\_GT(val1, val2)} & \texttt{val1 > val2}  \\ \hline
        \texttt{ASSERT\_GE(val1, val2)}  & \texttt{EXPECT\_GE(val1, val2)} & \texttt{val1 >= val2} \\ \hline
    \end{tabular}
\end{center}
\begin{itemize}
    \item \texttt{ASSERT\_EQ()} does pointer equality on pointers.
     If used on two C strings, it tests if they are in the same memory location, not if they have the same value.
    \item \texttt{ASSERT\_EQ(actual, expected)} is preferred to \texttt{ASSERT\_TRUE(actual == expected)}, since it tells you actual and expected's values on failure. 
    \item The arguments'evaluation order is undefined.
\end{itemize}

\subsection*{String Comparison}
The assertions in this group compare two \textbf{C strings}.\\
\textcolor{red}{If you want to compare two \textbf{string objects}, use \texttt{EXPECT\_EQ}, \texttt{EXPECT\_NE}, and etc instead.}
\begin{center}
    \footnotesize{
    \begin{tabular}{| c | c | c |}
        \hline
        \textbf{Fatal Assertion}           & \textbf{Nonfatal Assertion}     & \textbf{Verifies}     \\ \hline 
        \texttt{ASSERT\_STREQ(val1, val2)} & \texttt{EXPECT\_STREQ(val1, val2)} & The two strings have the same content \\ \hline
        \texttt{ASSERT\_STRNE(val1, val2)} & \texttt{EXPECT\_STRNE(val1, val2)} & The two strings have different contents \\ \hline
        \texttt{ASSERT\_STRCASEEQ(val1, val2)} & \texttt{EXPECT\_STRCASEEQ(val1, val2)} & The two strings have the same content, ignoring case \\ \hline
        \texttt{ASSERT\_STRCASENE(val1, val2)} & \texttt{EXPECT\_STRCASENE(val1, val2)} & The two strings have different contents, ignoring case \\ \hline
    \end{tabular}}
\end{center}
\begin{itemize}
    \item Note that "CASE" in an assertion name means that case is ignored.
    \item A NULL pointer and an empty string are considered different.
\end{itemize}

\section*{Test}
\begin{itemize}
    \item Use the \texttt{TEST()} macro to define and name a test function.
    \item In this function, along with any valid \textit{C++} statements you want to include, use 
        the various googletest assertions to check the values.
    \item The tests' results is determined by the assertions; if any assertions in the test fails, or
        if the tetst crashes, the entire test fails.
    \item \texttt{TEST()} arguments go from general to specific.
    \item A test's full name consists of its containing tests suite and its individual name.
    \item googletest groups the test results by test suites.
\end{itemize}

\section*{Test Fixtures}
To reuse the same configuration of objects for several different tests.
\begin{itemize}
    \item Derive a class from \texttt{::testing::Test}, start its body with \texttt{protected:}.
    \item Inside the class, declare any objects you plan to use.
    \item If necessary, write a default constructor or \textcolor{red}{\texttt{SetUp()}} function, to prepare the objects for each test.
    \item If necessray, write a default destructor or \texttt{TearDown()} function to release any resources allocated in \texttt{SetUp()}.
    \item When using a fixture, use \texttt{TEST\_F()} instead, as it allows you to access objects in the fixture.  
\end{itemize}

\begin{lstlisting}[language=C++]
    TEST_F(TestFixtureName, TestName){
        // test body
    }
\end{lstlisting}

Like \texttt{TEST()}, the first argument is the test suite name, but for \texttt{TEST\_F()} this must be the
name of the test fixture class.

\section*{Invoking the Tests}
\texttt{TEST()} and \texttt{TEST\_F()} implicitly register their tests with googletest.\\
Most users should not need to write their own main function and instead link with \texttt{gtest\_main} (as 
opposed to with \texttt{gtest}), which defines a suitable entry point.\\
The following is an example of fixture implementation:
\begin{lstlisting}[language=C++]
#include "this/package/foo.h"
#include "gtest/gtest.h"

namespace my {
namespace project {
namespace {

// The fixture for testing class Foo.
class FooTest : public ::testing::Test {
 protected:
  // You can remove any or all of the following functions if their bodies would
  // be empty.

  FooTest() {
     // You can do set-up work for each test here.
  }

  ~FooTest() override {
     // You can do clean-up work that doesn't throw exceptions here.
  }

  // If the constructor and destructor are not enough for setting up
  // and cleaning up each test, you can define the following methods:

  void SetUp() override {
     // Code here will be called immediately after the constructor (right
     // before each test).
  }

  void TearDown() override {
     // Code here will be called immediately after each test (right
     // before the destructor).
  }

  // Class members declared here can be used by all tests in the test suite
  // for Foo.
};

// Tests that the Foo::Bar() method does Abc.
TEST_F(FooTest, MethodBarDoesAbc) {
  const std::string input_filepath = "this/package/testdata/myinputfile.dat";
  const std::string output_filepath = "this/package/testdata/myoutputfile.dat";
  Foo f;
  EXPECT_EQ(f.Bar(input_filepath, output_filepath), 0);
}

// Tests that Foo does Xyz.
TEST_F(FooTest, DoesXyz) {
  // Exercises the Xyz feature of Foo.
}

}  // namespace
}  // namespace project
}  // namespace my

int main(int argc, char **argv) {
  ::testing::InitGoogleTest(&argc, argv);
  return RUN_ALL_TESTS();
}
\end{lstlisting}
\end{document}
